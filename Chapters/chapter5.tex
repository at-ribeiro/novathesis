%!TEX root = ../template.tex
%%%%%%%%%%%%%%%%%%%%%%%%%%%%%%%%%%%%%%%%%%%%%%%%%%%%%%%%%%%%%%%%%%%%
%% chapter5.tex
%% NOVA thesis document file
%%
%% Chapter with lots of dummy text
%%%%%%%%%%%%%%%%%%%%%%%%%%%%%%%%%%%%%%%%%%%%%%%%%%%%%%%%%%%%%%%%%%%%

\typeout{NT FILE chapter5.tex}%

\chapter{Evaluation}
\label{cha:evaluation}

In order to assess the developed system and answer the Research Questions proposed on Section~\ref{sec:objectives-and-researh-questions},
a user study was conducted. This chapter presents the design and results of this user study, exploring the findings and conclusions on 
the impact of the developed portal variants on usability, naturalness and spatial understanding.

\todo{add summary paragraph}

\section{Protocol}
\label{sec:protocol}

To ensure the validity of the results, the user study was always conducted in the same room, with an available tracking space of 
approximately 2.5m x 2.5m, providing conditions comparable to those available to most common \gls{VR} users~\cite{}. In addition, 
the same hardware was used across all participants to ensure consistency, being composed of: an \textit{Oculus Quest 3} \glsfirst{HMD} and 
a computer equipped with an NVIDIA GeForce RTX 3070 graphics card, 16 GB of RAM and a \todo{check CPU}.

Each participant experienced the four portal 
techniques—Traditional Portals, Movable Portals, Interactive Doors, and Revolving Doors—while performing the same tasks within 
the same \glsfirst{VE}. The only variation between participants was the order in which the portal techniques were presented, which 
followed a Latin Square design to minimize bias.

\subsection{Procedure}
\label{sec:procedure}

The experimental session commenced by asking the participant to read the informed consent presented in \todo{add appendix}. After reading, 
agreeing and signing the consent form, users were asked to fill a pre-session Virtual Reality Sickness Questionnaire (VRSQ), available in 
\todo{Add VRSQ annex}. Finally, before starting the session, participants also filled a characterization questionnaire that collected 
information about age, gender, education, experience in VR and video games, and sight problems.

After filling these initial forms, the experiment started with a brief of the context and instructions needed for the study given by the 
researcher. The participant was instructed on how to walk around the environment, how to use their virtual hands and 
information regarding the tasks they were asked to perform.  With the instructions provided, the participant started wearing the 
\gls{HMD} and was free explore to explore the \gls{VE} and complete the aforementioned tasks.

The \gls{VE} depicted a museum, structured in four thematic sections, each composed of four rooms populated with exhibition items. Every 
exhibit included an associated quiz or curious fact, designed to encourage engagement and support spatial memory. Each section differed 
from the others by theme, visual appearance, and the portal technique used to connect the rooms. Additionally, the starting room of each 
section featured a uniquely colored floor to help participants reorient themselves when asked to return to the starting point 
(Figure~\todo{add fig}).

Within each section, participants completed three tasks in sequence:  
1. \textbf{Exploration task:} freely exploring the rooms using the assigned portal technique, aiming to see and respond to every exposition items' quiz. After exploring the whole section, users were asked to return to the starting room.  
2. \textbf{Spatial memory task:} pointing from the starting room toward an object located in a previously visited room. To reduce memory bias, participants were allowed to revisit the object before returning to the designated location.  
3. \textbf{Subjective evaluation:} answering two Likert-scale questions (7-point scale) assessing naturalness and ease of use. These questions were presented inside the \gls{VE} to avoid breaks in presence.  

Throughout the session, the experimenter observed and recorded notable behaviors such as hesitations, verbal cues (e.g., “How do I...”), 
or repeated attempts to interact with portals, as objective indicators of usability.  

After completing all four sections, participants removed the \gls{HMD} and filled a post-session questionnaire. This included a post-session 
Virtual Reality Sickness Questionnaire (VRSQ), am Igroup Presence Questionnaire (IPQ), an Immersive Experience Questionnaire (IEQ), 
additional Likert-scale items assessing overall usability, a ranking of the four portal techniques by preference, open-ended questions on perceived 
strengths and weaknesses, and demographic data. Each session lasted approximately 30-45 minutes, depending on the participant's exploration speed.


\subsection{Data Collection}
\label{sec:data}



\section{Results}
\label{sec:results}

