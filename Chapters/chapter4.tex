%!TEX root = ../template.tex
%%%%%%%%%%%%%%%%%%%%%%%%%%%%%%%%%%%%%%%%%%%%%%%%%%%%%%%%%%%%%%%%%%%%
%% chapter4.tex
%% NOVA thesis document file
%%
%% Chapter with lots of dummy text
%%%%%%%%%%%%%%%%%%%%%%%%%%%%%%%%%%%%%%%%%%%%%%%%%%%%%%%%%%%%%%%%%%%%

\typeout{NT FILE chapter4.tex}%

\chapter{Implementation}
\label{cha:implementation}

Having established and presented the requirements and design choices, this chapter details the implementation of the concepts 
defined in Chapter~\ref{cha:analysis} using the \textit{Unity Game Engine}. The base of the developed system is the \glsfirst{VR} application 
responsible for rendering the \glsfirst{VE} constructed in \textit{Unity}, accurately processing and displaying the results of user 
interaction during the \gls{VR} experience. The \textit{Oculus Quest~3} were the chosen \glsfirst{HMD} to test and conduct this experience, 
however, the application resorts to \textit{OpenXR} libraries to further abstract the application, in order to make it compatible with most 
\glspl{HMD} with hand-tracking capabilities. Several asset \textit{XR} packages were utilized, streamlining \gls{VR} hardware integration, 
input handling and interaction systems. These packages will be addressed according to their use throughout this chapter.

\todo{Add summary}

\section{Player}
\label{sec:player}

\section{Portals}
\label{sec:portals}

\subsection{Traditional Portals}
\label{sec:trad-portals}

\subsection{Movable Portals}
\label{sec:mov-portals}

\subsection{Interactive Door}
\label{sec:interactive-door}

\subsection{Revolving Door}
\label{sec:rev-door}

\section{Tasks}
\label{sec:tasks}

\subsection{Exposition Items}
\label{sec:expo-items}

\subsection{Usability Questionnaire}
\label{sec:questionnaire}

\subsection{Pointing Task}
\label{sec:pointing-task}

\section{Summary}
\label{sec:summary}