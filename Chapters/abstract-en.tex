%!TEX root = ../template.tex
%%%%%%%%%%%%%%%%%%%%%%%%%%%%%%%%%%%%%%%%%%%%%%%%%%%%%%%%%%%%%%%%%%%%
%% abstract-en.tex
%% NOVA thesis document file
%%
%% Abstract in English([^%]*)
%%%%%%%%%%%%%%%%%%%%%%%%%%%%%%%%%%%%%%%%%%%%%%%%%%%%%%%%%%%%%%%%%%%%

\typeout{NT FILE abstract-en.tex}%

The purpose of this thesis is to improve how \gls{VR} users can navigate a large virtual room, whilst in a small physical space. 
The objective is to study and implement overt and subtle navigation redirection techniques, contributing to the research 
of immersive \gls{VR} navigation in restricted physical spaces using Non-Euclidean Spaces and \gls{RDW}.

\gls{VR} applications that grant users the ability to navigate in a very large \gls{VE} are often limited by the physical space in which they are 
used in. Thus, solutions for this problem have been proposed through the form of different locomotion techniques. Controllers provide 
a simple solution to this problem, by allowing users to navigate virtual worlds while standing still, 
however controller-based approaches have shown 
to be a less immersive solution for this problem compared to walking. 
By employing natural physical motion, \glsfirst{RDW} techniques have shown to be a more immersive solution, 
yet these often demand a larger tracking space to achieve its full immersive potential.

We intend to provide two alternative redirection techniques, Turn-and-Place and Hyperbolic Rooms, that allow navigation in constricted physical spaces, while prioritizing immersion, 
through the use of space subdivision for reorientation and hyperbolic spaces, respectively. 
These techniques may be used in \gls{VR} applications that prioritize immersion, such as those of entertainment and education, thus 
the implementations of these techniques will thematically follow those of a virtual museum experience.

The techniques will be evaluated through user studies, to validate their efficacy and usability, as well as a comparison study between them, 
to identify each of their strong points and limitations.

\keywords{
  Virtual Reality \and
  Redirected Walking \and
  Non-Euclidean Spaces \and
  Navigation \and
  Human-Computer Interaction
}