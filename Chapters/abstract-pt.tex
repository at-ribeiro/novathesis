%!TEX root = ../template.tex
%%%%%%%%%%%%%%%%%%%%%%%%%%%%%%%%%%%%%%%%%%%%%%%%%%%%%%%%%%%%%%%%%%%%
%% abstract-pt.tex
%% NOVA thesis document file
%%
%% Abstract in Portuguese
%%%%%%%%%%%%%%%%%%%%%%%%%%%%%%%%%%%%%%%%%%%%%%%%%%%%%%%%%%%%%%%%%%%%

\typeout{NT FILE abstract-pt.tex}%

O propósito desta dissertação é aprimorar a forma como os utilizadores de Realidade Virtual(RV) podem navegar num ambiente virtual de grandes dimensões, estando confinados a um espaço físico reduzido.
O objetivo consiste em estudar e implementar técnicas de locomoção com base em redirecionamento, tanto explícito como subtil, contribuindo assim para a investigação
de métodos de navegação imersiva em RV em espaços físicos limitados, recorrendo a Espaços Não Euclidianos e a técnicas de \textit{Redirected Walking}.

Aplicações de RV que permitem aos utilizadores explorar um ambiente virtual de grande escala encontram-se, frequentemente, 
limitadas pelas dimensões do espaço físico onde são utilizadas.
Por esta razão, têm sido propostas soluções para este problema através de diferentes técnicas de locomoção. Comandos oferecem
uma solução simples, permitindo que os utilizadores naveguem em ambientes virtuais enquanto permanecem imóveis,
contudo, as abordagens baseadas em comandos demonstraram ser menos imersivas quando comparadas com a locomoção por caminhada.
Ao empregar movimentos físicos naturais, as técnicas de \textit{Redirected Walking} têm-se revelado uma solução mais imersiva,
embora, frequentemente, exijam um espaço de rastreamento mais amplo para atingir o seu pleno potencial imersivo.

Propõe-se a implementação de duas técnicas alternativas de redirecionamento: Turn-and-Place (Virar e Posicionar) e 
Hyperbolic Rooms (Salas Hiperbólicas), que permitem a navegação em espaços físicos restritos, priorizando imersão,
através da subdivisão de espaços para reorientação e da utilização de geometrias hiperbólicas, respetivamente.
Estas técnicas poderão ser aplicadas em cenários de RV que privilegiem a imersão, tais como os da área do entretenimento e da educação, pelo que
as suas implementações nesta dissertação seguirão uma temática alinhada com uma experiência de um museu virtual.

As técnicas serão avaliadas através de estudos com utilizadores, com o intuito de validar a sua eficácia e usabilidade, 
assim como de um estudo comparativo entre ambas, de modo a identificar os pontos fortes e as limitações de cada uma.

\textbf{Palavras-chave:} Realidade Virtual, Redirected Walking, Espaços Não Euclidianos, Navegação, Interação Pessoa-Máquina