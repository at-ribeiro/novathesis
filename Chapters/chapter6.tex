%!TEX root = ../template.tex
%%%%%%%%%%%%%%%%%%%%%%%%%%%%%%%%%%%%%%%%%%%%%%%%%%%%%%%%%%%%%%%%%%%%
%% chapter6.tex
%% NOVA thesis document file
%%
%%%%%%%%%%%%%%%%%%%%%%%%%%%%%%%%%%%%%%%%%%%%%%%%%%%%%%%%%%%%%%%%%%%%

\typeout{NT FILE chapter6.tex}%

\chapter{Conclusion}
\label{cha:conclusion}

\glsfirst{VR} allows users to have immersive experiences in virtual worlds that defy the laws of our reality. 
These experiences can range from entertainment applications such as games and narrative experiences, to educational tools, training simulators, 
and therapeutic applications.
However, despite the inherent immersive capabilities of \gls{VR}, studies show that different interaction methods within a \glsfirst{VE} have 
effects on immersion and other aspects of \glsfirst{UX}. Given this, it is crucial to ponder and choose between different interaction techniques 
when designing and developing \gls{VR} applications. Among these techniques, navigation plays a central role, 
as it directly influences presence and usability.

When comparing common navigation techniques, such as teleport and joystick, natural walking has been
shown to be more intuitive and immersive. However, natural walking is inherently limited by the physical space
available to \gls{VR} users, which is typically about 2.4 x 2.2 meters for common \gls{VR} users.
To overcome these constraints, researchers have explored spatial manipulation techniques such as Redirected Walking, Impossible Spaces, 
and Portals, which expand the perceived environment while still relying on physical walking within a \gls{VE}.
Portals, in particular, let users move between distant locations in a visually continuous way, maintaining orientation cues and
spatial context. Despite their benefits, traditional portal implementations tend to
introduce practical limitations. To avoid collisions when turning through a portal, designers have to reserve part of the physical play
area for that purpose or place the portals at a distance from the limits of the \gls{VE}, wasting physical space or
creating unnatural blank spaces. Additionally, the visual presentation and interaction design of such portals often appear unnatural
or disconnected from real-world architectural elements, reducing spatial coherence.

This thesis addresses these identified limitations, by designing, implementing and studying four different \gls{VR} locomotion techniques 
based on natural walking through portals. From this work, three alternative portal designs were proposed and studied: 
\textit{Movable Portals}, which allow users to reposition portals, giving them control over spatial placement and offering 
a richer preview of the next room; \textit{Interactive Doors}, which mimic the everyday action of opening a real door, combining 
intuitive interaction with improved spatial continuity; and \textit{Revolving Doors}, which enable continuous transitions between rooms, 
providing a sense of flow and spatial coherence. These portal alternatives were studied and compared with a \textit{Traditional Portal} 
baseline design, providing insights on the impacts of the different portal variants on \gls{UX}

\section{Conclusions}
\label{sec:conclusions}

To address the identified limitations of portals, three research questions were defined and presented on 
Section~\ref{sec:objectives-and-researh-questions}. To guide this research a 31 participant user study was conducted, comparing the four portal 
variants in what concerned  impact on spatial understanding, usability, naturalness and user preference.

Regarding the first research question - \textbf{(Q1) What are the effects of a dynamic preview in regard to users' sense of space?} -
the study indicates that dynamic previews influence spatial judgments. 
The three proposed portal variants addressed the preview limitations of Traditional Portals when placed close to walls to save space, through the 
use of dynamic previews that provide a continuous perspective of space before the user transitions through them.
The study involved a pointing task, in order to evaluate participants' spatial understanding by
asking them to point towards the location of a previously visited museum object after traversing a portal.
From the results of this task, it was possible to conclude that user behavior changed from Traditional Portals to the three implemented portals.
Under the context of Traditional Portals, users consistently pointed towards the object in its overlapped position, indicating that users' perception 
of space was not continuous.
For the other portal variants, a distinct change of approach was identified, as most answers were divided into one of two clusters, either the 
position of the object in an overlapped space, or the position of the object would be in a continuous environment, using the preview as a reference. 
This indicates that portal variants with continuous or adjustable previews can enhance spatial continuity. 

Regarding the second research question - \textbf{(Q2) What elements of \glsfirst{UX} do users prioritize when interacting with these navigation techniques?} - 
the study highlighted that participants valued ease-of-use, familiarity and naturalness in interaction.
This conclusion was obtained through the results of the portal ranking users provided at the end of the experience, as well as their justifications 
for their ranking.
Traditional Portals were the highest ranked, considered the most efficient and easy to use, largely due to their simplicity, requiring only that users walk through them. 
Interactive Doors also scored positively, with participants recognizing their resemblance to real-world metaphors and appreciating the sense of natural interaction. 
Conversely, Movable Portals and Revolving Doors introduced a steeper learning curve: some participants hesitated or struggled during first use, 
though others justified their preference by valuing the additional control of Movable Portals or the continuous redirection pf Revolving Doors. 
Overall, ease of use and perceived naturalness emerged as dominant \gls{UX} priorities, 
even when participants recognized the potential benefits of more complex designs.

Regarding the third research question - \textbf{(Q3) What are the strong points for each of these techniques, in regard to \glsfirst{VE} navigation?} - 
the results showed that each portal design presents unique advantages. 
Although the already identified limitations of traditional portal implementations were again critiqued during the study by participants, 
the Traditional Portals stood out for their efficiency and straightforwardness, being well suited for contexts where clarity and 
minimal cognitive load are essential. 
Movable Portals empowered users by allowing them to adapt spatial transitions according to their preferences,
though at the cost of higher cognitive effort. 
Interactive Doors balanced immersion with intuitiveness, offering a natural metaphor for transitioning between rooms. 
Revolving Doors provided the strongest impression of spatial continuity once mastered, but demanded greater familiarity, as it was not 
as intuitive as the other alternatives. 

This research demonstrates that alternative portal designs can mitigate the limitations of traditional implementations by 
enriching spatial previews and naturalness of interaction, though often at the cost of usability. Designers of VR applications should 
therefore carefully balance these trade-offs, selecting or combining portal designs according to the intended \gls{UX}. Traditional Portals 
are ideal for environments where ease of use and quick adoption are priorities, as they require less user intervention.
Since having the portal detached from a wall can feel unnatural in indoor settings such as real-world buildings, this design may be
better suited for outdoor, less realistic, or fictional scenarios. Interactive Doors are well suited for applications that benefit from
natural interactions and spatial continuity in indoor environments, such as museums or architectural walkthroughs. Movable Portals
are most appropriate in scenarios where users need control over space and previews, such as exploration-based games or creative
design environments. Revolving Doors can be advantageous in experiences emphasizing continuous transitions and a sense of flow,
such as immersive storytelling or puzzle-based VR worlds. Selecting the right portal design according to the intended use can help
maximize both usability and overall user experience. 


\section{Future Work}
\label{sec:future-work}

This work researched and addressed the limitations of portals, however, further work on this topic could contribute to the further research of 
\gls{VR} navigation in small physical environments through \glsfirst{RDW} techniques reliant on portals.
Alternative research possibilities for this topic include:

\begin{itemize}
    \item \textbf{Extended evaluation across use cases} - Investigate how the portal variants perform in different \gls{VR} scenarios, 
    such as collaborative tasks, storytelling experiences, or larger-scale environments, to assess generalizability and context-dependent 
    advantages.

    \item \textbf{Refinement and innovation of portal designs} - Design and test new portal variants or hybrid techniques that 
    address usability and spatial perception limitations, minimizing the usability costs introduced by the proposed techniques.

    \item \textbf{Longitudinal studies} - Explore user adaptation and learning effects over multiple sessions to understand how 
    familiarity with portal-based locomotion influences performance, presence, and cybersickness.

    \item \textbf{Integration with other locomotion methods} - Combine portal-based techniques with methods such as redirected walking,
    joystick locomotion, or teleportation to evaluate hybrid approaches in constrained physical spaces.

\end{itemize}

