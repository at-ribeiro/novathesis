%!TEX root = ../template.tex
%%%%%%%%%%%%%%%%%%%%%%%%%%%%%%%%%%%%%%%%%%%%%%%%%%%%%%%%%%%%%%%%%%%
%% chapter1.tex
%% NOVA thesis document file
%% Chapter with introduction
%%%%%%%%%%%%%%%%%%%%%%%%%%%%%%%%%%%%%%%%%%%%%%%%%%%%%%%%%%%%%%%%%%%

\typeout{NT FILE chapter1.tex}%

\chapter{Introduction}     
\label{cha:introduction}

\prependtographicspath{{Chapters/Figures/Covers/}}
\newcommand{\Overleaf}{\href{https://www.overleaf.com?r=f5160636&rm=d&rs=b}{Overleaf}}

% epigraph configuration
\epigraphfontsize{\small\itshape}
\setlength\epigraphwidth{12.5cm}
\setlength\epigraphrule{0pt}


\glsfirst{VR} is an immersive technology, with the capacity to transport users into interactive computer-generated 3D \glsfirst{VE}. Unrestricted 
by the laws that bind the real world, \glspl{VE} can bend or dismiss these rules to create virtual spaces that are impossible to experience otherwise. 
As such, \glspl{VE} can be larger than the physical environment \gls{VR} users have available while navigating these virtual worlds. 

To resolve this real-world limitation, different locomotion techniques have been developed. The most common techniques rely on the use of artificial means for 
locomotion, such as controllers, yet these have shown to be a less immersive solution than naturally walking in a physical environment. Therefore,
\gls{VR} applications that prioritize immersion often rely on the use of motion-based techniques such as \glsfirst{RDW}, to allow users to physically 
walk in their restricted space safely, without reaching the ends of their tracking space. 



\section{Motivation}
\label{sec:Motivation}

The implementation and exploration of immersive \glspl{VE} has evolved with the development of \gls{VR} and its wide 
range of locomotion techniques. 
These techniques influence dimensions of \gls{UX}, such as efficiency, usability, immersion, comfort, and accessibility, and are, 
therefore, used in different contexts and situations~\cite{Boletsis2019}.

One of the main challenges of \gls{VR} locomotion is the physical limitations users face when navigating a \gls{VE}.
The most common techniques rely on the use of controllers for locomotion, such as teleportation and joystick-based movement~\cite{Coomer2018}, 
due to the fact that controllers do not warrant users' movement and there is a familiarity with them from the video-game scene.
Although artificial techniques have shown to be an efficient and appropriate method of locomotion in \gls{VR}, the lack of a 
walking motion makes these techniques less suitable for use-cases in which immersion is prioritized, as a natural walking motion 
provides higher feelings of presence by mirroring how humans walk in the real world~\cite{Nilsson2018}.

Locomotion techniques that rely on users to walk in their physical environment have shown their benefits in immersion and presence. Motion-based techniques 
include the use of gestures that emulate walking, such as Walking-in-Place or Arm-Swinging, or the use of artificial 
devices that counter-act users' movements while walking, keeping them in place, such as Treadmills. These techniques have 
shown to be a preferable solution for higher feelings of immersion than controllers~\cite{Nilsson2018}.

\gls{RDW} refers to a group of techniques for locomotion in \gls{VR} that allow users to naturally walk through their \gls{VE}, by keeping  
users inside their available space through a variety of different redirection methods~\cite{8255772}. Although most commonly associated with methods such as 
rotational gains, that play with the imperceptibility of human senses to slowly direct users to their available space,
\gls{RDW} techniques can employ redirection with more overt strategies for 
safety-risking cases or others in which imperceptibility is kept by discreetly changing the \gls{VE} in real-time, 
while the user is navigating it. Despite being a promising solution that enables walking in restricted spaces, 
it usually needs to be employed in use-cases in which users' tracking spaces are approximately 5 x 5 meters, 
where gains remain imperceptible to users~\cite{Razzaque2001, Steinicke2010}. However, the average available physical area for 
\gls{VR} users is approximately 2.4 x 2.2 meters \cite{Azmandian2015, Liu2018b}. 

Given the possibilities of shaping rules of reality in \gls{VR}, a form of implementing \gls{RDW} is through the use o Non-Euclidean Spaces. 
By employing non-Euclidean properties it is possible to map larger \glspl{VE} into a limited tracking space, by using overlapping-architectures
that break the laws of Euclidean Geometry~\cite{Suma2012,Vasylevska2017} or by mapping Euclidean coordinates into a hyperbolic plane~\cite{Pisani2019, Rebelo2022}. 

Portals are commonly used to connect multiple sections of a self-overlapping architectural layout, typically functioning as doorways that allow users 
to seamlessly travel between locations within a \gls{VE} through bidirectionally paired connections~\cite{Freitag2014,Lochner2021,Liu2018b}.
The seamless transition from one section to another is achieved through a stereoscopically rendered plane, enabling users to preview the connected 
space as if they were already positioned at the destination, with the appropriate transformations of location and rotation 
relative to the portal~\cite{Lochner2021,Jaksties2022}.

However, portals as a means of locomotion have shown its limitations in previous studies~\cite{Rebelo2024,Rebelo2025}. 
To prevent users from leaving the limits of their physical tracking space, portals must be placed within a certain distance from the boundaries 
of the corresponding \gls{VE}. This has an impact on the portal's preview, as the closer the portal is to said limits, 
the less space will be observable through the preview, since the rendering of the portal view is dependent on the position of the portal. 
Additionally, positioning portals within a certain distance from these limits hinders the amount of available space in an interior \gls{VE}, 
as well as its naturalness, due to the unreal-like appearance of a floating door frame away from a wall. 

With this context in mind, the goal of this dissertation is to contribute to the research of \gls{RDW} in small physical spaces, through the design,
implementation and study of portal alternatives, aiming to counter-act the previously mentioned limitations.
This work pretends to enhance navigation experiences in \gls{VR}, making virtual experiences more accessible, particularly in home settings, 
where users typically have no more than 2.4~x~2.2 meters of space to walk.


\section{Objectives and Research Questions}
\label{sec:objectives-and-researh-questions}

The main objective of this dissertation is to contribute to the research of \gls{RDW} in limited physical spaces through the design, implementation and 
study of portal variants for \gls{RDW} in small tracking spaces. To complete this main objective, smaller particular goals have been 
identified for conducting this research:
\todo{Talvez um obj sobre criar a VE? Parece-me pouco.}

\begin{itemize}
    \item \textbf{Design and Implement the four alternative portal techniques}:
    Conceive and accomplish each of the techniques, addressing each of the most common portal limitations found in previous studies.

    \item \textbf{Conduct user studies to evaluate the techniques}: 
    Conduct empirical studies with participants in order to assess the usability, effectiveness, and overall user experience of the proposed portal techniques.
    
\end{itemize}

In order to better structure this research, the following three research questions are proposed:

\begin{itemize}
    \item \textbf{Q1 - Do portals with a dynamic preview help convey a continuous sense of space for users?} - 
    
    \item \textbf{Q2 - What elements of \gls{UX} do users prioritize when interacting with these navigation techniques?} - 
    
    \item \textbf{Q3 - What are the strong points for each of these techniques, in regard to \gls{VE} navigation?} - 
    
\end{itemize}

\section{Solution Overview}
\label{sec:solution-overview}

Our solution relied on the implementation of three distinct portal variants that address the limitations of \gls{VR} navigation using portals. 
These variants were compared with an implementation of traditional portals through a user study, taking a museum tour as a use-case, 
where participants explored a museum \gls{VE} using each of the developed portal variants, bound by a physical space of 2.5m~x~2.5m.

A more detailed description of the design choices and implementation of each of the portal variants is present throughout the document, however, a general 
definition of each of the techniques are as follows:

\begin{enumerate}
    \item \textbf{Traditional Portals (Baseline)} - A simple framed portal with a stereoscopically rendered preview, always placed within a minimum 
    walkable distance from the limits of the \gls{VE}. To use this portal a user must solely walk through the frame.
    
    \item \textbf{Movable Portals} - A framed portal initially placed at the wall that users can move by grabbing the portal through a handle-like
    affordance present in the frame. When grabbed users may choose the placement of the portal inside a valid area by letting go of the portal. 
    When the portal is against the wall the preview is reversed so that the preview may be continuous to the next room. Placing the portal makes 
    it rotate, reversing the preview. To use this portal the user must grab it, place it and then walk through it.
    
    \item \textbf{Interactive Doors} - A door-shaped portal, complete with a door-frame and handle, placed against the wall. The door changes its 
    appearance according to user's proximity: opaque when near, transparent when distant. Given this, the portal is visible from a distance, displaying a
    continuous preview that becomes less visible when users approach it. The door can be opened when fully opaque by grabbing the handle in the door-frame, 
    with the portal following the door-frame with the motion. After opened the user may teleport to their destination by walking through the opened door.
     
    \item \textbf{Revolving Doors} - A compartment in the \gls{VE} with a revolving door, that similarly to Interactive Doors, changes its opacity 
    according to user distance. From afar the door is transparent, making the portal's continuous preview visible, but when approached it becomes 
    opaque and interactable. To use the portal the user must push the door and perform a 180-degree turn.
\end{enumerate}

A user study was conducted using these portal variants, comparing various usability factors between the baseline of traditional portals and our variants, 
as well as between each other. This study aimed to gather data on each of the portal's impact on user preference, usability, spatial understanding,
presence and comfort. 

\section{Document Structure}
\label{sec:doc-struct}

\todo{change according to the final document}

Following this section, the main goal of the \nameref{cha:related_work} chapter is to explore the literature on this topic. The chapter starts by presenting
\glspl{HMD} and devices used in \gls{VR}, as well as how Navigation occurs on \glspl{VE}, followed by an exploration of several different 
common types of \gls{VR} Locomotion techniques, with one of them being an extensive exploration of \gls{RDW} techniques. Lastly, this chapter 
finishes by addressing two types of Non-Euclidean Spaces in \gls{VR}: \nameref{sec:impossible-spaces} and \nameref{sec:hyperbolic-spaces}. 

The \nameref{cha:plan-and-analysis} chapter describes the proposed solutions in detail, and addresses the user studies to be conducted. 
This chapter also includes a section on the preliminary work accomplished to the present date and finishes by addressing the planned work to 
be done during the dissertation.
