%!TEX root = ../template.tex
%%%%%%%%%%%%%%%%%%%%%%%%%%%%%%%%%%%%%%%%%%%%%%%%%%%%%%%%%%%%%%%%%%%
%% chapter1.tex
%% NOVA thesis document file
%% Chapter with introduction
%%%%%%%%%%%%%%%%%%%%%%%%%%%%%%%%%%%%%%%%%%%%%%%%%%%%%%%%%%%%%%%%%%%

\typeout{NT FILE chapter1.tex}%

\chapter{Introduction}     
\label{cha:introduction}

\prependtographicspath{{Chapters/Figures/Covers/}}

% epigraph configuration
\epigraphfontsize{\small\itshape}
\setlength\epigraphwidth{12.5cm}
\setlength\epigraphrule{0pt}

[TODO: Completely outdated]

The implementation and exploration of safe, expansive and immersive \glspl{VE} has evolved with the development of \gls{VR} and its wide range of navigation techniques. 
These techniques influence several dimensions of \gls{UX}, such as efficiency, usability, immersion, comfort, and accessibility, and are, therefore, used in different contexts and situations.

In small physical spaces, the most commonly used navigation techniques rely on joystick-based movement and teleportation, due to the physical contraints of the environment. 
By cutting the use of Natural Walking, these techniques prove to be less immersive and unrealistic, as they trade the realism of walking elements for ease-of-use.

To address these limitations, new navigation paradigms have been developed. Some of these resource to the use of Impossible Spaces and 
Non-Euclidean Geometry to provide users the ability to naturally walk through extensive \glspl{VE}, even whithin restricted physical spaces. 
By manipulating spatial perception and geometry, these techniques create the illusion of larger virtual spaces, enabling more intuitive and immersive navigation experiences.


\section{Motivation}

\section{Related Questions}

Q1-Is it worthwhile to explore navigation through non-Stride techniques whilst in a constricted physical space?

Q2-Does using hyperbolic strides instead of linear ones lead to increased disorientation and cybersickness?

Q3-Do hyperbolic spaces effectively convey more control to the user compared to spaces with linear strides?

% Other \newcommand and configuration settings retained as necessary.

\newcommand{\Overleaf}{\href{https://www.overleaf.com?r=f5160636&rm=d&rs=b}{Overleaf}}