%!TEX root = ../template.tex
%%%%%%%%%%%%%%%%%%%%%%%%%%%%%%%%%%%%%%%%%%%%%%%%%%%%%%%%%%%%%%%%%%%
%% chapter1.tex
%% NOVA thesis document file
%% Chapter with introduction
%%%%%%%%%%%%%%%%%%%%%%%%%%%%%%%%%%%%%%%%%%%%%%%%%%%%%%%%%%%%%%%%%%%

\typeout{NT FILE chapter1.tex}%

\chapter{Introduction}     
\label{cha:introduction}

\prependtographicspath{{Chapters/Figures/Covers/}}

% epigraph configuration
\epigraphfontsize{\small\itshape}
\setlength\epigraphwidth{12.5cm}
\setlength\epigraphrule{0pt}

\section{Motivation}
Joysticks and teleportation have been made standard methods of locomotion in VR Navigation, but  some users experience motion sickness or lose their sense of immersion whilst recurring to these navigation methods. With this said, the goal of this dissertation is to explore new experimental ways to provide users with different methods of navigation in Virtual Spaces, while minimizing and maximizing the ammount of motion sickness and immersion, respectively.

To achieve this we use PCG algorithms and Impossible Spaces(like portals) to map the expansive Virtual Space into the limited real space that an user has to move, that, in conjuction with other techniques like Redirected Walking, create the possibility for a new VR Navigation paradigm the excludes the need for controllers.

AUTHOR'S NOTE: De acordo com a última reunião o foco não vai ser tanto em PCG. É preciso alterar esta descrição e redirecionar o foco. 

\newcommand{\Overleaf}{\href{https://www.overleaf.com?r=f5160636&rm=d&rs=b}{Overleaf}}

% Other \newcommand and configuration settings retained as necessary.
